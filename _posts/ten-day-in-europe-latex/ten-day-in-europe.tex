\documentclass{ctexart}
\usepackage{fancyhdr}
\usepackage[bottom]{footmisc}
\usepackage[margin=1.25in]{geometry}
\usepackage[hidelinks]{hyperref}
\renewcommand{\headrulewidth}{0pt}
\usepackage{bookmark}

\newcommand{\IMG}[1]{
	\begin{center}
	\makebox[\textwidth][c]{\includegraphics[width=\paperwidth]{#1}}
	\end{center}
}

\newcommand{\IMGT}[2]{
	\begin{center}
	\makebox[\textwidth][c]{\includegraphics[width=\paperwidth]{#1}}
	\makebox[\textwidth][r]{\footnotesize#2}
	\end{center}
}

\newcommand{\SECTION}[1]{
	\section*{#1}\addcontentsline{toc}{0}{#1}
}

\pagestyle{fancy}
\rhead{\thepage}
% \pagenumbering{gobble}
\graphicspath{{imgs/}}
\setlength{\parskip}{1em}

\title{欧洲十日}
\author{Hanlin}
\date{\today}

\begin{document}

\makeatletter
{
	\noindent
	\Huge\@title
	\newline
	\newline
}
\makeatother

\IMGT{eu0.jpg}{去机场的路上}

我早早的就到了机场,等待另外两个同行的伙伴,国际航班的等候大厅里大多数都是中国大妈,她们熙熙攘攘地簇拥在举着旅行团旗子的导游身旁。我找了个空位坐下来,旁边的大妈蹲在地上正匆匆忙忙地把照片贴在重要的文件上,一旁则散落着从她刚打开的行李箱的衣物,她看上去既兴奋又慌张。

过了一会儿一位年纪稍大的外国大叔走了过来,用眼神和手势询问我旁边的座位是不是没人,我笑着点了点头。大叔坐下后拿出手机翻看,无意间看到是在查看法兰克福的天气。

\IMG{eu1.jpg}

在飞机上只睡了一觉,醒来时拉开遮光板,外面一半漆黑一半深蓝,交接的地方发着深黄色的光,我拿出手机想把这一刻照下来,无奈呈现在手机上的影像比不上半点眼前所看到的。飞机上的时间比想象中的要过的快,坐我旁边的德国大叔个头比我大出了一半还多,看得出他坐着十分不舒服,不停的喝着青岛啤酒。

\IMGT{eu2.jpg}{法兰克福的高楼}

快到法兰克福时天开始的亮了,从飞机上可以看到身后的阳光,飞机在往西边飞,太阳则从东南边升起,窗户的景象几乎没有变化,好像飞机的速度和日出的速度达到了某种平衡。

从飞机上看到的法兰克福很平,高楼都集中在一起,多少显得有些突兀,城市蒙着一层黎明时暗暗的蓝色的光。

出境没有想象中的那么顺利,我们排在一个旅行团的前面,团员基本都是较大岁数的大妈大叔,导游告诉他们准备好相关的资料,我们才想起我们并没有刻意准备任何出境的资料,并且身上带的欧元现金并不多(我带了 20 欧)而从导游口中了解到跟团的每人每天至少要带 100 欧左右。

幸好我准备了行程单,租车凭证,以及在 Airbnb 上定的所有住宿,并且把它们都打印了出来,主要是怕在旅行中没有网络的情况下能派上用场,没想到在出境的时候用上了, 即使这样也没有那么顺利,检查官还问了返程的航班信息,并且查看了邀请信,在看 Airbnb 的资料时他还笑了笑(估计是觉得我们的行程太赶了吧)

入境后就很顺利了,也没有查看行李箱携带的物品,出门就看见王双脚盘坐在椅子上,对着笔记本在浏览着什么。

机场外的天气比预计的要冷一些,阳光很强;我们顺利的取到事前预约好的车,就近找了一个超市买了点零食和水果还有最重要的红牛。

早午饭也就在超市买了点面包和肉饼解决了,期间的谈话一直逃不出「他们这边的某某是这样的噢」的句式。

\IMGT{eu4.jpg}{第一顿饭}

吃完饭准备出发时却发生了一件很尴尬的事情:不知道如何把车挂到倒档上。最后不得不求助刚好来超市买东西的德国女司机,结果是用的力不够大,女司机不停的笑,记得当时好像是说「从来没有想到在德国会帮男人的解决开车的问题」

\IMGT{eu5.jpg}{租到的车}

旅行的第一天就这样开始了

\newpage
\SECTION{DAY I}

虽然刚到德国,但第一个目的地却是布拉格,从法兰克福到布拉格的距离几乎等于横穿了整个德国。

德国开车跟国内一样,方向盘在左侧,靠右行驶;开始时多少有些紧张的,即使有些路标在来之前都仔细看了一遍,但是开在路上又是另外一回事,路况也不熟悉。后来开到高速路上就好多了,主要没有那么多限制,而关于德国高速不限速的迷思也解开了:部分高速路段不限速。


除了王,一路上我们都在不停感叹:

「天气太好了吧!」

「你看那朵云!」

「这些车太守规矩了!」

「风车!」

风车其实是风能发动机。

\IMG{eu7.jpg}

捷克的高速是需要收费的,不分行驶的路段收费自然也没有收费站,高速票分为十天、一个月、一年,购买了贴在挡风玻璃上境内所有的高速都可以通行,费用比国内便宜很多,十天的票 100 RMB 不到,不过在边境购买的高速票要比官方给出的价格高 5 欧左右。

\IMGT{eu8.jpg}{A6 高速上德国与捷克交界处高速票购买的招牌}

在捷克开车可以明显地感觉到没有在德国舒服,大型货车频繁超车,而就在我们去布拉格的高速路上还目睹了正前方一辆轿车失控,在路中间转了好几圈最后停在了右侧的缓冲带上,幸好没有什么大问题。又想起在出发前一个月就在布拉格订好的民宿,临近出发前一周房东才说要去其他地方工作了,之前订的房子没有办法入住了;我们又才临时换了住宿,而当时已经没有什么好的选择了,这个意外也直接造成了我们必须把车停在城区里,花费了不少的停车费(750 捷克币一晚)。虽然这两件事并没有直接的关联。

\IMGT{eu9.jpg}{斯柯达是捷克的汽车品牌}

布拉格城市里有轨电车随处可见

\IMG{eu9.1.jpg}

即便是在狭窄的主干道上

\IMG{eu9.2.jpg}

也有相对较旧的电车

\IMG{eu10.jpg}

开着车在城市里穿梭还不会感觉特别不一样,把着停好,走在布拉格的路上才会明显感觉到不同,建筑的不同,交通工具的不同,人种的不同。即使开了一天的车,我依旧不感觉疲倦,走在路上每一个角落都吸引着你,或许是路程第一天的缘故兴奋之情难以言表。

\IMG{eu11.jpg}

我们沿着伏尔塔瓦尔河往北走,路过一个不大的广场,一群小朋友正在演出,吸引了不少人围观。

\IMG{eu12.jpg}

另一侧有个小女孩好奇地在弹弄着钢琴。

\IMGT{eu12.5.jpg}{远处可以看到布拉格城堡和查理大桥}

穿过一座桥

\IMGT{eu13.jpg}{列侬墙}

再穿过一个公园,去看了看列侬墙,大多是年轻人,周围停满了车,除了一个贩卖拍照的小伙子就看不见其他的商贩了。

\newpage
\IMGT{eu14.jpg}{查理大桥上的雕塑}

而查理大桥上两旁则排满了商贩,贩卖着各式各样的纪念品,又怀表、有画作,还有乐队在桥上现场演奏,地上摆放着他们自己的专辑的专辑。

\IMG{eu15.jpg}

在去布拉格老城广场的路上遇到了猫头鹰,不过它好像并不开心。

布拉格广场的有几堆表演的团队,有些是卖艺的有些则是单纯的表演。有个大叔在广场表演各种魔术,他的打扮以及装备让我想起了电影《魔术师》里面的情景,突然想到如果把他放到他穿着打扮的那个年代他就是那个年的 Freelancer 吧。

\IMG{eu16.jpg}

我们爬上了钟楼,可以看到整个布拉格城市,随手一拍就是明信片。

\IMG{eu17.jpg}

穿过泰恩教堂,去一家评价不错的餐厅,但被告知已经没有位置了。又去了附近另外一家没有什么人的,晚上不用开车,我也喝了啤酒,可能是期待太高或是没有去到计划的那家餐厅的原因,并没觉得特别好喝。

\newpage
\IMG{eu18.jpg}

住宿的地方另外一户门上印着\textbf{IN FILM}不知道是真的在拍电影或者仅仅是个标签。

\newpage
\SECTION{DAY II}

\IMGT{eu24.jpg}{城堡中的小道}

第二天我们几个都起来的很早,大概是因为时差的原因加上旅行的兴奋,当天的计划去布拉格城堡,去的时候几乎没有人,停车场收费员还没来上班,周围的餐厅也都没有开门。

\IMG{eu20.jpg}

入口附近碰见穿着中世纪服装的壮汉,不知道他是在城堡里上班的还是 COSPLAY,之后在城堡里也再也没有遇见到他。

\newpage
圣维特大教堂应该是我见过的第一座教堂,刚走出长廊拱门巨大的正门压在眼前,实在震撼。

\IMGT{eu22.jpg}{圣维特大教堂背面}

教堂背后的飞拱\footnote{\url{https://en.wikipedia.org/wiki/Flying_buttress}}

\IMGT{eu23.jpg}{圣维特大教堂内部的窗子}

教堂内部很高,因为来的早参观的人也并不是很多,少数人在里面祷告,而最漂亮的是光线从外面通过彩色的玻璃窗射进教堂内部又印到墙上的效果,每一块窗户玻璃的颜色都不相同,每一块窗户两侧的壁画也不相同。

\IMGT{eu25.jpg}{Restaurace U veverky}

吃过午饭,前往维也纳,一路上比较顺利,奥地利也是需要购买高速票的,收费方式和价格跟捷克的差不多,我们准备把票贴在挡风玻璃上的时候,旁边一个瑞士的车主主动拿着胶水过来给我们说奥地利的高速票不好贴,并帮我们把高速票的背面涂了一些胶水。

\IMGT{eu26.jpg}{奥地利某个加油站对面的油菜花}

一路上风景依旧很美,但我们都没有发出刚到德国时那种感叹了。

开到奥地利的某个路口远远的看见几个年轻人在招手,应该是要搭车,要不是我们车上已经坐满了,我肯定会停下来搭他们,这件事情对我来说多少有些遗憾。

\IMGT{eu27.jpg}{奥地利市区的红绿灯}

不知道是因为我开的不好还是奥地利的司机太暴躁了,一路下来在维也纳市区里开车给了我很大的压力,红绿灯很多间隔时间也异常的短,只要你稍微慢一点后面的车就不停的按喇叭。市区里其他车辆开的飞快,摩托车引擎的声音也异常的大,周围的人都习以为常.

\IMG{eu28.jpg}

奥地利的建筑给人的感觉要比布拉格的庄严对称许多。

\newpage
\IMGT{eu29.jpg}{金色大厅}

晚上同行的两个朋友去了金色大厅,我跟王则背着在布拉格还没有喝完的啤酒在维也纳市区闲逛。

\IMGT{eu30.jpg}{傍晚的查理教堂}

我们坐在查理教堂\footnote{\url{https://en.wikipedia.org/wiki/Karlskirche}}水池边上打算把捷克的啤酒喝完再去其他地方,看着天色慢慢从天蓝变成深蓝色;在我们旁边来了七八个人提着很多批萨坐了下来,看着我们也饿了,计划这用书包里剩下的啤酒给他们换一些批萨,但被他们拒绝了;旁边又来了一个女孩,坐在离我们不远的地方打开素描本对着教堂认真的画画。我们差不多准备走的时候之前那堆吃批萨的朋友把他们剩下的半个批萨给了我们,并示意不用给他们啤酒。

\IMGT{eu31.jpg}{奥地利地铁}

从地铁出来时天已经完全黑了,去了圣斯特凡大教堂,不知道是不是晚上的原因已经没有圣维特大教堂给带来的那种冲击,教堂里面的人大多也都不是游客。

\IMG{eu32.jpg}

我们穿过霍夫堡宫,往维也纳艺术史博物馆走,那是我们明天的目的地,博物馆早已关门了,但灯还是一直亮着的,坐在的大门前拿出之前买的花生,继续解决在布拉格没有喝完的啤酒。离我们比较远的地方坐着一个女的,王说她好像在抽大麻,因为他闻到了味道。

\newpage
\IMGT{eu33.jpg}{躺在艺术史博物馆大门前的我}

一路上喝的实在是有点多,尿急。周围又找不到厕所,我们趁着夜色借着酒劲在草丛堆里解决了。

\newpage
\SECTION{DAY III}

\IMGT{eu34.jpg}{维也纳周末早上的街道}

这天依旧起来的很早,睡得比前一天晚上好些,应该是星期天的原因街道上只有零零散散的几个人,我们各自躺在床上用手机分享着《Before Sunrise》\footnote{\url{http://www.imdb.com/title/tt0112471/}}里杰西和塞琳娜去过的地方,并计划着如果时间路程合适要去哪些地方。

\IMGT{eu35.jpg}{游行的队伍}

那天刚好是五一国际劳动节,在市政厅附近有个党派正在准备宣传。简单在街口解决了早饭,周围的人多了起来,各个路口站着苦脸的警察,我们才意识到是有游行,他们举着旗帜,嘴上也不喊些什么就这样在平时汽车行驶的道路上有序的前进,各式各样的人都来参加了,有穿着正式严肃的中年大叔,有带着搜救犬的消防员,有演奏乐器的,有开着拖拉机的,还有小朋友骑着三轮脚踏车跟在队伍旁边。

\IMGT{eu36.jpg}{维也纳艺术史博物馆}

这次的行程我最期待的就是去艺术史博物馆

\IMGT{eu37.jpg}{博物馆内部的天花板}

一进去就被震撼到,特别是到了油画区,二十米左右层高的房间里挂满了油画,眼睛根本不知道往哪看,连「哇」这种感叹的声音都发不出来,里面像一个信息密度极高的迷宫。

\newpage
\IMGT{eu38.jpg}{Café Sperl}

从博物馆出来我们去了杰西和塞琳娜在《Before Sunrise》里玩电话游戏的那家咖啡馆(Café Sperl\footnote{\url{https://www.youtube.com/watch?v=qO6fmMzopw8}})吃午饭,坐在红色发暗的沙发上很容易想起电影里的场景,生活在这里的人真是太幸福了。

下一个目的地是威尼斯,因为里维也纳距离较远,又不想开夜路,我把当晚的住宿定在了斯洛文尼亚的一个山。一路上大家都比较疲惫,兴奋的劲也在第三天耗完,路上的风景依旧如画,下着大雨,经过某座山时起了大雾,那种完全看不到前面任何情况的大雾,而周围的车子依旧开的比较快,我们不得不在一个加油站休息了一会才继续赶路

\IMGT{eu39.jpg}{意大利与斯洛文尼亚的边境}

一直走到意大利的边境我们才又兴奋了起来。

刚过意大利边境,就近找了一个小村庄吃饭,停车时遇见一直小猫,那时天还下着毛毛雨。

或许是因为饿了,那天是我吃过最好吃的批萨,可惜还要继续开车,没办法喝啤酒。

吃完饭天差不多快黑了,路越来越窄,慢慢地看得见道路两旁零零散散堆积的雪,驾驶的时候也感觉的到路面的湿滑,海拔不断升高。一路上也几乎没有其他的车辆,同行的伙伴甚至看到了路边一个小木屋里吊着的干尸,最终到达是天已经几乎完全黑了,但可以模糊地望见对面的雪山。

我们住在一栋背靠雪山两层的房子里,出门除了一条双向单车道就是悬崖,停好车我跟王抑制不住兴奋的情绪即使下着小雨也搬了两把椅子坐在门前望着雪山喝着啤酒。

\IMG{eu41.0.jpg}

住宿的房子第一层的层高比较矮,大概只有两米的样子,客厅里布置感觉很老但又感觉恰如其分,电视旁边放着给客人留言的笔记本,我们几个也混乱写了些上去。

\newpage
\SECTION{DAY IV}

第二天一早起来,看到的景象是这样的:

\IMG{eu41.1.jpg}

昨晚我们就住在其中的一栋房子里,因为只住了一晚离开时多少有些舍不得,并计划着下一次一定来多住几天。

雨忽大忽小,一路上查着查威尼斯的天气预报,抱着侥幸希望到达时雨就停了。

意大利的高速跟国内的一样是分段收费,费用也比捷克和奥地利贵了不少。

到达威尼斯的时候还是下着小雨。

\IMGT{eu42.jpg}{雨中的威尼斯}

我们穿梭在威尼斯里狭窄的石板路上,因为下雨的缘故路上的行人很少,偶尔遇到打伞的爷爷和奶奶都主动把雨伞偏到一旁让我们先过去。

吃过午饭雨渐渐小了人也多了起来,游客明显多了。

威尼斯所有的交通工具都是船,出租车是船,公交车也是船只不过大了许多。一路走到圣马可广场,成群结对的鸽子一点也不怕人,只要你有食物它们就会飞到你周围来进食,把面包放在手上它们甚至飞到你手臂上来,当然我们还尝试了把面包放在头顶,它们依然会来。遇到有人往地上扔一大块面包时就会吸引体型比鸽子大许多的海鸥把整块面包叼走。

\IMG{eu43.jpg}

为了省钱,我们去坐了两欧的贡多拉,只是从岸的这边划到岸的那边,直线距离还不到五十米。

直到我们往回走的时候天才开始放晴,威尼斯也变的好看起来。

\IMG{eu44.jpg}

从威尼斯到米兰的路程不近,定民宿时选择了一个靠近米兰的小镇 Iseo

\IMGT{eu45.jpg}{Lago d'Iseo}

Iseo 对我们所有人来说绝对是一个意外的收获,非常漂亮的湖,远处可以望见对面山脚下的小镇。除了我们镇上没有什么游客,非常安静。

刚到的时候询问了一对大叔停车的问题,随后因为房东联系不上,一直在镇上乱转,而每次走过那个露天咖啡馆都会看见那对帮助我们的大叔,他们就一直在那咖啡厅坐着,真的可以坐一下午。

晚上随意选择了一下靠近湖岸的餐厅,刚好还坐临湖的窗边,看着日落吃了在欧洲最好吃的一顿。

\newpage
\SECTION{DAY V}

\IMGT{eu46.jpg}{米兰地铁}

米兰因为有 ZTL\footnote{\url{https://en.wikipedia.org/wiki/Zona_a_traffico_limitato}}(限行区) 的限制开车非常不方便,我们把车停在了一个最远的地铁站(Sesto 1 Maggio Fs)坐地铁前往米兰大教堂。

\IMGT{eu47.jpg}{米兰大教堂顶上的雕塑}

米兰大教堂要比布拉格的圣维特大教堂震撼一些,特别的是还可以爬到教堂的顶上去,而教堂的内部更是阔气很多,宽敞(教堂内部的一个角落甚至放了部用于维修的机械装置)的大厅显得人很少。

\IMGT{eu48.jpg}{Galleria Vittorio Emanuele II}

从教堂出来,去了旁边的拱廊街\footnote{\url{https://en.wikipedia.org/wiki/Galleria_Vittorio_Emanuele_II}}帮朋友买东西,一来是自己并不熟悉要买的东西(多是化妆品),而来是让朋友一路当翻译总感觉有些不妥,购买的过程也并不那么顺利,还不停通过微信语音交流,站在售货员的面前等待着一条一条的语音,还不时把商品拍下来发过去,实在有些尴尬。

\IMGT{eu49.jpg}{Galleria Vittorio Emanuele II}

下午计划是去圣玛丽亚修道院看看达芬奇《最后的晚餐》的真迹,之前也在网上查了相关资料需要预订但是一直找不到入口,只好来了再碰碰运气,果然没有预订是没有办法看的。

\IMGT{eu50.jpg}{玛丽亚修道院}

这样时间就多出来了一些,想着可以早点去瑞士,当我们就近找吃饭的地方时第一家老板就告诉我们「除了快餐,全米兰就没有五点就开的餐馆」,我们只好往回走,刚好看见一家批萨外卖点,进去时发现师傅是亚洲人的面孔,交谈后发现就是中国人,就这样我们在米兰用中文点了批萨,师傅很自豪的说他们做的批萨是米兰最薄的。

吃完坐地铁取到车就前往瑞士了,路上瑞士牌照的车越来越多也越来越好。

下了 A2 高速在双向单车道上走,道路两旁都是田野,窄窄的路偶尔能遇见一辆车,正前方能远远地看到小镇的样子,穿过了小镇进入盘山公路,一路上我尽量把车速控制在限速上限,但身后紧跟的车越来越多,看到一个可以靠边休息的地方,我赶紧把车停了下来让他们先过去,而再等我启动企图跟在他们后面时已经看不见他们的尾灯了。

山谷里不时听见跑车发动机轰鸣的声音,瑞士的老司机实在太野,山路还开那么快。更有趣的是我们把车停在另一个靠边的地方拍照时,突然有一辆越野车从山下开了下来迅速地在我们面前停住,问我们「有没有大麻」或是「问我们要不要大麻」无果后笑着扔下一句「no grass no party」一脚油门就开走了。

虽然我们出发的很早,但找到住宿位置时天色已经很暗,因为当晚是住在山上的一个蒙古包里。刚到时非常幸运的看见了一只小鹿,停好车时它已经跑开了。房东很热情提前帮我们准备好了木材,教会我们如何使用火炉并向我们确认明天多早起来,以便给我们准备早餐。

蒙古包的顶上有一个透明的圆形玻璃,看上去好像还有点凸透镜的样子,把屋里的灯都关掉可以透过它看见天空的星星。唯一有点不方便的是蒙古包内没有厕所,要洗漱需要在外面单独的厕所进行,而当我们整理好行李,点燃火炉后天已经完全黑了,还刮起了风,昼夜温差大,要出去洗个澡还是需要点勇气。

我跟王到车上拿东西回来的路上时候甚至听到像狼一样的叫声。

\IMG{eu51.jpg}

当晚在蒙古包里就着肉片、花生喝着啤酒,外面不时一阵阵狂风刮过,每次听到反狂风过,我们都不约而同的凝住,确认真是的风(最开始的甚至有地震的感觉)。

\newpage
\SECTION{DAY VI}

\IMGT{eu52.jpg}{蒙古包外的瑞士早餐}

早上起来的很早,往山上小小的逛了一圈回来时房东已经准备好了早餐:热牛奶,果汁,奶酪,各种风味的果酱和各种形状的烤面包,放在木制的篮子里,那感觉太好了。

早上的阳光很强,吃早饭的时候也要带着墨镜,抬头就可以看见雪山,不时有飞机飞过,在天上划过一条细细的线。

吃完早饭,把蒙古包里的东西都收拾干净,给房东打了声招呼我们就往 Lucerne 走了,下山的路比昨天上山的路好开一些,身后也没有紧跟很多车。穿过曾经世界最长的隧道(圣哥达公路隧道\footnote{\url{https://en.wikipedia.org/wiki/Gotthard_Road_Tunnel}},全长16.32千米,现在是世界上第三长的公路隧道),风景越来越好。

\IMGT{eu53.jpg}{开往 Rigi 的火车}

绕过卢塞恩湖,在离 Rigi 上最近的一个镇上(Arth)把车停好,并买了最近一班的火车。

\IMGT{eu54.jpg}{火车尾部}

火车上的人并不多,都是游客;因为很陡的原因火车行进的也很慢;一路从黄绿相间的田野开到了白雪皑皑的山上,车内的温度也能明显的感觉到在不断下降。

\IMG{eu55.jpg}

山顶并没有那么冷,太阳很大,周围一片白色;山顶上有一块大石头上面写着「峨眉山」,石头下面有块圆形的大理石,并有一个箭头指向东边写着从 Rigi 山到峨眉山的直线距离;山上还有一个气象站,有时发出嗡嗡的声音。

\IMG{eu56.jpg}

山上全是白雪,往北边可以看到楚格湖,南边则是卢塞恩湖和雪山,视野异常的好。有时可以看到飞机平行的从远处飞过,偶尔也可以看到几个跳伞的人。

\newpage
\IMGT{eu57.jpg}{不时有飞机飞过}

我们没法在山上玩很久,因为晚上必须赶回德国,下山的火车上太困了,不自觉的睡了一觉。

回到德国就又可以体验不限速的高速公路了,这次的状态显然比刚到德国时好多了,习惯了路上汽车的速度。

晚上是在 Oenburg 附近的一个农家里住的,门口关着一只很大的狗,到的时候房东不在,他的儿子接待了我们,院子很大,两栋房子一栋应该时单独来做民宿,另一栋则是房东家自己住;收拾好行李去了主人家推荐的一下餐厅。

刚进餐厅就傻了,屋里全是上年纪的人,那一瞬间我们望着他们,他们也望着我们,可能就连服务员也没有换过神来,我们点好菜,不停的讨论这家餐厅实在太 local 以及哪一桌的哪个人在盯着我们。

菜上来了,分量很足,我们像在国内吃中餐一样互相分着吃,这一行为又吸引了旁边那桌人的目光,特别是其中一个老爷爷,毫不忌讳的一直盯着我们看,他应该时觉得太神奇了:四个年轻的亚洲面孔背着书包在一家全是老头聚餐的地方互相分着彼此的食物。当时我们没有意识到我们已经变成了「外国人」

晚上只喝了一杯啤酒,因为太累的缘故,自己能够感觉到是有点醉的,我万分注意的把车开回了住处,所幸没有出事。

\newpage
\SECTION{DAY VII}

当天去了欧洲公园(Europa-Park\footnote{\url{https://en.wikipedia.org/wiki/Europa-Park}}),1975年建成的主题公园至今仍是德国最大,全欧洲第二(第一是巴黎的迪士尼公园)的主题公园。

人比想象的多,是我们在欧洲见过人最密集的一天,自然所有的项目都是需要排队,排队的时间还不短,总之那天全去坐了过山车,木头的过山车,旋转的过山车,室内的过山车,以及三次 Silver Star\footnote{\url{https://en.wikipedia.org/wiki/Silver_Star_(roller_coaster)}}(直到 2012 年还是欧洲最高的过山车,最快时速可达 130 km/h )

当晚我们依旧去了昨天那家全是老爷爷的餐厅,这次进去是跟昨天的情景差不多,有一桌坐满了刚登山完的老爷爷(他们的背包装备都堆放在门口)

\newpage
\SECTION{DAY VIII}

当天我们计划去巴登巴登(Baden-Baden)去泡温泉晚些时候再赶往斯图加特,而从奥芬堡(Offenburg)到巴登巴登我们故意绕道到东边一点穿过黑森林(Schwarzwald)的 B500 公路,一路上的风景完全不输瑞士,周围密密麻麻高大的树,路面有些湿滑,应该是还没有被晒干露水。

开到处较高的休息站,看见有人在玩滑翔伞,我们停了下来。

\IMG{eu58.jpg}

坐在木凳子上等着看他们下一次滑翔,令人感到意外的是他们可以滑的比起飞时还要高,并且可以控制降落到起飞的位置,我们还猜想他们只能往下滑然后再坐车上来。

下山的路依旧比上山好开许多,速度不自觉的提上来,弯道却没有减少,一路上不停有摩托车队从车后穿过。

\newpage
到巴登巴登后去了一家混浴温泉,刚进去时多少有些尴尬,温泉里指明了要每一到工序,具体已经记不清楚了,大概是先洗净身体,分别在温度由低到高的两个桑拿室呆上足够的分钟数,然后在几个不同功能、温度的温泉里泡上一会,最后到只有几度的水池中尽可能的多呆几秒,我摸了一下水温实在太冷,放弃了最后一道工序。洗完过后如果有需要还提供休息的躺椅和自助的泡茶。要不是约定好时间简直不想走。

从巴登巴登到斯图加特高速有很多处不限速,一度把车开到了 200km/h 左右,再多踩一点油门车就快飘起来了。

\IMGT{eu59.jpg}{199 km/h}

\newpage
\SECTION{DAY IX}

到奔驰博物馆时才知道刚好赶上十周年庆典。

博物馆内部结构是旋转上升的,进馆时乘坐电梯到最顶层,然后像盘山公路一样旋转着往下参观,一路上的陈设按照时间顺序往下旋转延伸,从 1886 年第一辆现代汽车到如今混合动力跑车,期间穿插着各种特定类别的展设,包括各类名人的坐驾、政府公共部门使用的功能行汽车以及校车,甚至可以坐到校车内部亲自感受。

\IMGT{eu60.jpg}{物馆内部的电梯}

内部的两部电梯极具上个世纪对未来的想象,电梯外部光滑没有棱角,全身银色,但并没有强烈的金属质感,横向的开口相对整个电梯显得很窄,却异常契合,甚至让人想到《2001 太空漫游》里的那艘飞船(Discovery One\footnote{\url{https://en.wikipedia.org/wiki/Discovery_One}})。从电梯里可以通过开口看到正对面投影到墙体上播放的记录影像。

从博物馆出来发现有些道路被交警拦住了,后来看到街上穿着红白相间衣服的球迷,同行的朋友才意识到今天应该有斯图加特的比赛,查了一下果然如此,我们临时决定下午去看看,后来因为时间的关系放弃了。

中午经朋友建议去了一家叫 Alte Kanzlei 的餐厅吃午饭,餐厅旁边是一个很大的广场,草坪上稀稀疏疏坐着人,我们也找了个地方躺下,休息了一会儿就往法兰克福走了。

我们提前跟房东沟通好,预计下午六点钟到,并准备在他家自己烤烧烤吃,当我们到达时才五点过,敲了门没有人,就先去附近的超市买烧烤要用的木炭、食材还有啤酒。回来时房东已经在家了。

房东是一个德国人,但他却想去美国,客厅布置有吧台,吧台背后的架子上摆满了各式各样的酒,地上铺着像奶牛皮一样的地毯,屋里还有一只黑猫和一条很壮的金色大狗,刚进屋时不停摇尾巴。

晚上把行李都收拾好,在院子里烤着烧烤喝着啤酒看着夕阳慢慢的落下,这是我们在欧洲的最后一个晚上了,饭间的闲聊多少有些别离的感觉,至少没有那么兴奋了。

\newpage
\SECTION{DAY X}

当天早早的把油加满,顺利地把车还了。租车公司的人把我们送到了机场。飞机是晚上八点过的,我们准备去法兰克福市区逛逛,顺带买点东西带回去,万万没有想到那天是星期天也就是礼拜天,从地铁出来在繁华的法兰克福商业街上除了流浪汉见不到几个人,所有的商店都关着门,连餐馆开着的也不多。

穿过教堂,我们懒散的坐在美因河(Main)的长凳上,看着跑步的人从眼前穿过,消磨着时光,还去了有欧元标志的草坪上趟了一会,可能是这几天实在太累了,我记得我在草坪上睡了好一会,直到他们把我叫醒。

晚些时候我们去了人最多的火车站里的快餐店吃了肘子和土豆作为离开欧洲吃的最后一顿。

\IMG{eu61.jpg}

这趟旅行我们一共行驶了 3024 公里(刚好是从成都开到北京,再从北京开到上海的距离),去了六个国家,几乎每天都换一个城市过夜,所有的住宿都是在 Airbnb 上预订的,路线也几乎跟之前计划的一模一样,回想起来多少还是有些不可思议。

\fancyfoot[R]{\bfseries\thepage Hanlin by \LaTeX\\\today}

\end{document}
